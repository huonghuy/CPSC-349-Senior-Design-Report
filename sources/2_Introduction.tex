%Introduction Paragraph to Chapter
%LawnBot is an affordable automated lawnmower that is designed for your average homeowner. With a budget of \$500 we plan on competing with other automated lawnmowers that go for anywhere from \$900 - \$2000. We plan to compete with them by using cheaper exterior materials instead of metals but also using reliable hardware with equally as reliable software. The LawnBot will be setup with a controller that the user controls to control the robot so it records the data on where the user wants the LawnBot to cut the grass. This lawnmower will have different sensors to detect different items in front of it so it can avoid them. This project is aimed to have an affordable automated lawnmower that helps anyone who could be to busy or does not have the capabilities to cut their own grass. \par

\subsection{Weekly Overall Status}
Since the material list won't be obtained until late November, a wiring sketch has started that coexist with the project's main design. This procedure is perfect to pass on the time and have a realistic viewpoint on the project's implementation and design. This week we also discovered our project was accepted as a paper for the 2025 ASEE National Convention in Montreal, Canada. At least one of our team members and our advisor plan to represent our team, but we aim to have the entire team travel as it is a huge honor for us all.\par

\subsection{Challenges}
Some challenges we are facing is that it will be difficult to make any progress on the machine until we have our equipment. Another challenge we are going to face is submitting everything to ASEE on top of our current deadlines. ASEE requires a draft paper to be submitted by January 15th, 2025 of six pages without references. The sooner we have our equipment, the sooner we can begin testing and writing our documentation.
%Some issues our group is currently are the design and load distribution. After meeting with Dr.Patel, he brought up concerns over our amperage used in our current sketch. Our current design draws approximately 10 Amperes between our current fan choice and the other components. Our fan is currently drawing the most power at approximately 6.6 Amperes, which will take nearly a third of our battery within an hour by itself. Our group is currently recalculating the our power draws to make sure everything is correct. \par

%With our load distribution, we are looking into how to support the 12 Volt 15 Amp-hour battery we have purchased. The battery is currently comes in at 6.5 pounds and we will need to plan our unit around this. We also need to make sure that our motors and wheels will be able to support this weight without compromising the integrity of the unit. \par
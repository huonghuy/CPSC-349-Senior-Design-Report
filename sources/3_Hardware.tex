%Introduction Pragraph
The dummy kit has been purchased and built. Our main factor, for the kit, is the fact that it behaves like a vehicle but interacts with the environment. The way it does that is with a multi-directional (Horizontally) sonar (attached to servos) to replicate the front part of our project.\par

\begin{comment}
    \subsection{Material List}
    We have drafted a complete material list and are ready for unexpected errors/malfunctions that we may face during the design phase. The project's has taken a turn for the better by decreasing the its size by 1 ft of difference compared to our abstract build. We've removed the concept of using dual blades and applied a singular mower blade for less voltage usage and size decremental, making our project possible without having a crazy powerful battery. We've decided that our project won't have a GPS/RTK Base station, due to the fact that we can simply implement the module to the robot itself, and a 2D LIDAR due to our design choices, leading to less expenses.\par
    \begin{table}[H]
       \begin{tabular}{|c|c|} \hline 
       \centering
        \textbf{Item}& \textbf{Quantity}\\ \hline
           ESP32-WROOM-32& 1\\ \hline   
           L298N Motor Driver& 1\\ \hline
            Gear Motor 300 RPM 12V& 3\\ \hline 
           Voltage Regular (XL6009)&1\\ \hline 
            Jumper Wires (120 pc)&1\\ \hline 
            20 Gauge Hookup Wires&1\\ \hline 
            Fuse Holder&1\\ \hline 
            Drive Wheel&1\\ \hline 
            Caster Wheel&1\\ \hline 
            10" Cooling Fan&1\\ \hline 
            Ultrasonic Distance Sensor&8\\ \hline 
            GPS/RTK Module&1\\ \hline 
            Rain Sensor Module&1\\ \hline 
            Power Switch&1\\ \hline 
            Battery&1\\ \hline
            Wire Holder&2\\\hline
            Pixhawk PX4 Flight Controller&1\\\hline
        \end{tabular}
        \caption{Material List as of \today} 
        We can do either \today if we want to have it constantly updating or we can just make it October 21, 2024
    \label{tab:table1}
        
    \end{table}
\end{comment}

 
\subsection{Design}
During these last few days, we have been rethinking our main design. The ring design was a simple prototype design idea, and lacked originality when compared to other products on the market. Currently, the design is being reconsidered for a more optimal and unique design. We are not completely starting over, as the same issues with the prototype design (such as the vacuum) will still persist, and the solutions that we had created for these issues will still be remediated all the same. As for materials, we have decided to test print in PLA, and for the final product to use the ASA as planned. \par

\begin{comment}
    \begin{figure}[H]
        \centering
        \includegraphics[width = 0.8\textwidth]{root/Lawn_Bot_V1.jpg}
        %\includegraphics[width = 0.2\textwidth]{root/UB-Seal.eps}
        \caption{LawnBot Design (\today)} 
        \label{fig:LawnBot_Design_V1}
    \end{figure}
\end{comment}

\subsection{Testing Phase}
Realizing the material list's delivery date, we purchased a kit to test on, containing a similar microcontroller, to start the software and wiring testing/implementation. This will speed up the process into understanding the automation field of the project and start the testing phase of our project while helping us plan ahead once we get our actual equipment. \par